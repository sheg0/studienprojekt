\chapter{Einleitung}
\section{Motivation und Zielsetzung}
In einer Zeit, in der das digitale Studium stetig an Bedeutung gewinnt, stehen Studierende vor der Herausforderung, eine Vielzahl an Terminen und Materialien effizient zu verwalten. Handschriftliche Notizen sowie verstreute Kalendereinträge verursachen häufig Unübersichtlichkeit und Stress.\newline 
Mit der zunehmenden Digitalisierung steigt der Bedarf an eigenverantwortlicher Organisation des Lernalltags. Studien zeigen, dass Selbstorganisations- und Motivationsfähigkeiten zentrale Voraussetzungen für erfolgreiches Arbeiten und Lernen sind - insbesondere in agilen und digitalen Umgebungen \autocite{selbstorganisation}. Gerade für Studierende bedeutet das, geeignete Werkzeuge zu nutzen, um Termine, Inhalte und Notizen effizient zu strukturieren. Die im Rahmen dieser Arbeit entwickelte Web App adressiert genau diese Bedürfnisse. \newline \newline
Ziel ist es, eine Progressive Web App (PWA) zu entwickeln, die nicht nur Vorlesungspläne und Termine in einer übersichtlichen Oberfläche bündelt, sondern auch das Anlegen und Wiederfinden persönlicher Notizen erleichtert. Dabei stehen eine gebrauchstaugliche Bedienung, Offline-Funktionalität sowie eine nahtlose Synchronisierung über verschiedene Endgeräte hinweg im Fokus. Die Anwendung soll Studierende dabei unterstützen, ihren Studienalltag effizienter zu strukturieren und sich auf das Wesentliche - das Lernen - zu konzentrieren.

\section{Projektgegenstand}
Der Gegenstand dieses Projekts ist die Konzeption und Umsetzung einer PWA für Studierende, die drei Kernmodule integriert:
\begin{enumerate}
  \item \textbf{Vorlesungsmanagement}, das Stundenpläne visuell darstellt.
  \item \textbf{Kalender}, der Termine anzeigt.
  \item \textbf{Notizmodul}, das das Erstellen, Kategorisieren und Durchsuchen von persönlichen Lernnotizen ermöglicht.
\end{enumerate}
Die technologische Basis bildet ein ReactJS-Frontend, das PWA-Features nutzt, sowie ein in Node.js entwickeltes Backend, das in Docker-Containern läuft und eine MySQL-Datenbank verwaltet. Durch diese Architektur gewährleistet die Web App Skalierbarkeit, einfache Erweiterbarkeit und einen reibungslosen Betrieb in unterschiedlichen Hosting-Umgebungen.